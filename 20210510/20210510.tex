\documentclass[12pt]{jsarticle}

\usepackage{amsmath,amssymb,amsbsy,amscd,mathrsfs}
\usepackage{fancybox,ascmac}

\begin{document}

\noindent
rootが$\land$の3分木Tを考える.\\
rootの子を左から$a,b,c$とする,\\
この時,$d_0(a)=a_0 , \ d_1(a)=a$と表す.$b,c$も同様にして定める.\\
\begin{center}
\begin{tabular}{|c|c|c|c|c|}
\hline
     & \multicolumn{1}{c|}{アルゴリズム} & \multicolumn{3}{c|}{コスト} \\ \cline{1-5}
重み  & 探索する順番 & $(a,b,c)=(0,1,1)$ & (1,0,1) & (1,1,0) \\ \hline
$p_1$ & $a, b, c$ & $a_0$  & $a+b_0$  & $a+b+c_0$  \\ \hline
$p_2$ & $b, c, a$ & $a_0+b+c$  & $b_0$  & $b+c_0$  \\ \hline
$p_3$ & $c, a, b$ & $a_0+c $ & $a+b_0+c$  & $c_0 $ \\ \hline
$p_4$ & $b, a, c$ & $a_0+b$ & $b_0$  & $a+b+c_0$  \\ \hline
$p_5$ & $a, c, b$ & $a_0 $ & $a+b_0+c$ & $a+c_0$ \\ \hline
$p_6$ & $c, b, a$ & $a_0+b+c$  & $b_0+c$ & $c_0$  \\ \hline
\end{tabular}
\end{center}
$(a,b,c)=(0,1,1)$である時のコストを${\rm W_{011}}$,他同様にして${\rm W_{101}}$, ${\rm W_{110}}$とすると,\\\\
${\rm W_{011}}=a_0+(b+c)p_2+cp_3+bp_4+(b+c)p_6 \ \ \ \ \, (1.1)$\\
${\rm W_{101}}=ap_1+b_0+(a+c)p_3+(a+c)p_5+cp_6 \ \ \ \ (1.2)$\\
${\rm W_{110}}=(a+b)p_1+bp_2+c_0+(a+b)p_4+ap_5 \ \ \ \ (1.3)$\\\\

ここで、$a_0=b_0=c_0~かつ~a=b=cのときはd_0(T)=\Psi 'を示せている$\\\\

$proof$\\
背理法で示す.即ち, $\Psi' > d_0(T)$となる様な, $(p_1',p_2',\ldots,p_6')$が存在すると仮定.\\
この時, $(p_1',p_2',\ldots,p_6')$の下での,  ${\rm W}_{011}, {\rm W}_{101}, {\rm W}_{110}$を
${\rm W_{011}}',{\rm W_{101}}',{\rm W_{110}}'$と表すと, 以下のようになる.\\
${\rm W_{011}}'=a_0+a+a(p_2'+p_6')-a(p_1'+p_5')$\\
${\rm W_{101}}'=a_0+a+a(p_3'+p_5')-a(p_2'+p_4')$\\
${\rm W_{011}}'=a_0+a+a(p_1'+p_4')-a(p_3'+p_6')$\\
この時,  $\Psi' > d_0(T)$より,
\begin{center}
 $\Psi' >{\rm W_{011}}' $かつ$\Psi' > {\rm W_{101}}'$かつ $\Psi' >{\rm W_{110}}' $.
\end{center}
であることに注意すると, $\Psi' > d_0(T)$となる様な, $(p_1',p_2',\ldots,p_6')$が存在する為の必要十分条件は,
\begin{center}
 $(p_1'+p_5') >(p_2'+p_6') $かつ $(p_2'+p_4') >(p_3'+p_5') $かつ  $(p_3'+p_6') >(p_1'+p_4') $.
\end{center}
一方,
\begin{eqnarray*}
(p_1'+p_5') +(p_2'+p_4') +(p_3'+p_6')&>&(p_2'+p_6') +(p_3'+p_5')+(p_1'+p_4')\\
p_1'+p_2'+p_3'+p_4'+p_5'+p_6'&>&p_1'+p_2'+p_3'+p_4'+p_5'+p_6' \\
1&>&1
\end{eqnarray*}
これは矛盾.\\
よって, $d_0(T)=\Psi'$\\\\

\vspace{5mm}\\
次に、$a=b=c~のとき、d_0(T)=\Psi '~を示す$\\\\
$proof$\\
前回までに、$d^*_0(T)を$\\\\
$d^*_0(T)~:~{\rm W_{011}},~{\rm W_{101}},~{\rm W_{110}}~の3式が等しいRDAに限定したd_0(T)$\\\\
と定義したとき、$d^*_0(T)=\Psi '$~は示せている\\
ここで、$d^*_0(T)=d_0(T)$を示す\\
この等式を示すには、${\rm W_{011}},~{\rm W_{101}},~{\rm W_{110}}~の3式が等しくなるようなRDAの存在を言えばいい$\\
\begin{itembox}[l]{主張1}
${\rm W_{011}}={\rm W_{101}}\Leftrightarrow $
$p_2=\frac{1}{a+b+c}\{-(a+b)p_4+cp_5-(a+b)p_6\}+\frac{1}{a+b+c}(b_0+a-a_0)$
\end{itembox}
\begin{itembox}[l]{主張2}
${\rm W_{101}}={\rm W_{110}}\Leftrightarrow p_3=\frac{1}{a+b+c}\{ap_4-(b+c)p_5-(b+c)p_6\}+\frac{1}{a+b+c}(c_0+b-b_0)$
\end{itembox}
\begin{itembox}[l]{主張3}
${\rm W_{011}}={\rm W_{110}}\Leftrightarrow p_1=\frac{1}{a+b+c}\{-(a+c)p_4-(a+c)p_5+bp_6\}+\frac{1}{a+b+c}(a_0+c-c_0)$
\end{itembox}\\
以上の主張より、\\
$p_1=\frac{1}{3}\{-ap_4-ap_5+p_6\}+\frac{1}{3a}(a_0+a-c_0)$\\
$p_2=\frac{1}{3}\{-ap_4+p_5-ap_6\}+\frac{1}{3a}(b_0+a-a_0)$\\
$p_3=\frac{1}{3}\{p_4-ap_5-ap_6\}+\frac{1}{3a}(c_0+a-b_0)$\\
を満たすように、$(p_1,\ldots,p_6)の組み合わせをとれば、{\rm W_{011}}={\rm W_{101}}={\rm W_{110}}~が成り立つ$\\
$ため、{\rm W_{011}},~{\rm W_{101}},~{\rm W_{110}}~の3式が等しくなるようなRDAの存在を言えた$\\
以上から、$d^*_0(T)=d_0(T)$を示せた。\\
よって$a=b=c~のときd_0(T)=\Psi '$\\\\


$a_0=b_0=c_0のときd_0(T)=\Psi 'を示す$\\\\
$proof$\\\\
前回までの内容より、$以下の主張4が言えており$
\begin{itembox}[l]{主張4}
${\rm W_{011}}={\rm W_{101}}={\rm W_{110}} $
{\scriptsize  \[\Leftrightarrow
\left(
\begin{array}{c}
p_1 \\
p_2 \\
p_3
\end{array}
\right)
=
\frac{1}{a+b+c}
\left(
\begin{array}{ccc}
-(a+c) & -(a+c) & b \\
-(a+b)  & c  & -(a+b) \\
a   & -(b+c) & -(b+c)
\end{array}
\right)
\left(
\begin{array}{c}
p_4 \\
p_5 \\
p_6
\end{array}
\right)
+
\frac{1}{a+b+c}
\left(
\begin{array}{c}
a_0+c-c_0 \\
b_0+a-a_0 \\
c_0+b-b_0
\end{array}
\right)
\]}
\end{itembox}\\\\
$a_0=b_0=c_0$のとき、以下の主張4'が成り立つ\\\\
\begin{itembox}[l]{主張4'}
${\rm W_{011}}={\rm W_{101}}={\rm W_{110}} $
{\scriptsize  \[\Leftrightarrow
\left(
\begin{array}{c}
p_1 \\
p_2 \\
p_3
\end{array}
\right)
=
\frac{1}{a+b+c}
\left(
\begin{array}{ccc}
-(a+c) & -(a+c) & b \\
-(a+b)  & c  & -(a+b) \\
a   & -(b+c) & -(b+c)
\end{array}
\right)
\left(
\begin{array}{c}
p_4 \\
p_5 \\
p_6
\end{array}
\right)
+
\frac{1}{a+b+c}
\left(
\begin{array}{c}
c \\
a \\
b
\end{array}
\right)
\]}
\end{itembox}\\\\
ここで\\\\
$p_1=\frac{1}{a+b+c}\{-(a+c)p_4-(a+c)p_5+bp_6\}+\frac{c}{a+b+c}$\\
$p_2=\frac{1}{a+b+c}\{-(a+b)p_4+cp_5-(a+b)p_6\}+\frac{a}{a+b+c}$\\
$p_3=\frac{1}{a+b+c}\{ap_4-(b+c)p_5-(b+c)p_6\}+\frac{b}{a+b+c}$\\\\
を満たすように$(p_1,\ldots,p_6)の組み合わせをとれば$\\
${\rm W_{011}}={\rm W_{101}}={\rm W_{110}} $を満たし、\\
$a_0=b_0=c_0のときd_0(T)=\Psi 'を示せた$\\


\end{document}
