\documentclass[11pt]{jarticle} %jarticleクラスを使用,文字サイズを11ポイントに設定
\setlength{\textheight}{23cm} %本文領域の高さ(縦幅)を23cmに設定
\setlength{\textwidth}{16cm} %本文領域の横幅を16cmに設定
\setlength{\oddsidemargin}{0cm} %左端の余白をデフォルトから0cmに設定
\setlength{\topmargin}{0cm} %上端の余白をデフォルトから0cmに設定
\renewcommand{\baselinestretch}{1.2} %行間をデフォルトの1.2倍に設定
\title{n分木の均衡値} %タイトルを記入
\author{栗田 亮也~~~~清水 泰良\\(東京都立大学 理学研究科 数理科学専攻)} %著者名を記入
\date{令和3年5月10日} %日付を記入

\usepackage{amsmath,amssymb,amsbsy,amscd,mathrsfs}
\usepackage{fancybox,ascmac}

\begin{document}


\maketitle %この位置にタイトルを作成


\noindent

\section{定義} %セクションを作成

\begin{center}

  $balanced~tree~T_nの部分木を左からt_1,~t_2,\cdots,~t_n$\\
  $t_1,~t_2,\cdots,~t_nのrootが~0~をとるコストを左から、a_1,~a_2,\cdots,~a_n$\\
  $t_1,~t_2,\cdots,~t_nのrootが~1~をとるコストを左から、b_1,~b_2,\cdots,~b_n$\\
  \vspace{3mm}
  ${\rm W_{1}}$~=~${\rm W_{011\cdots11}}$\\
  ${\rm W_{2}}$~=~${\rm W_{101\cdots11}}$\\
  ~\vdots~~~~~~~~~~~\vdots~~~~~~\\
  ${\rm W_{n-1}}$~=~${\rm W_{111\cdots01}}$~~~~\\
  ${\rm W_{n}}$~=~${\rm W_{111\cdots10}}$\\
  \vspace{3mm}
  調べ方は$n!$通りあり、$p_1,~p_2,\cdots,~p_m~~(m~=~n!)と書ける$\\
  \vspace{3mm}
  $T_nの均衡値d_0(T_n)を~\Psi_n~と表す$
\end{center}

\newpage

\section{証明} %セクションを作成

\begin{itembox}[l]{定理}
  $任意の整数i,~j~(i\ne j,~1\leqq i,j \leqq n)に対して、~ a_i \leqq b_i + a_j ~~~ かつ ~~~ a_j \leqq a_i + b_j ~を満たすとき$
  \begin{center}
    $d_0(T_n) ~ = ~ \displaystyle\frac{\displaystyle\sum_{i=1}^{n}a_ib_i~+~\displaystyle\sum_{i=1,j=1}^{n}a_ib_j}{\displaystyle\sum_{i=1}^{n} b_i} ~ = ~ \Psi_n $
  \end{center}
\end{itembox}\\

\underline{$n=2$}\\

こちらに関しては、さっくす と うぃるだぁそんが示しており\\

\begin{center}
  $d_0(T_2) = \displaystyle\frac{a_1b_1+a_2b_2+b_1b_2}{b_1+b_2} = \Psi_2$
\end{center}

\newpage

\underline{$n=3について考察する$}\\
\vspace{2mm}\\
$n=3のとき、~p_1,~\ldots,~p_6~を以下のように定める$

\vspace{3mm}

\begin{center}
\begin{tabular}{|c|c|c|c|c|}
\hline
     & \multicolumn{1}{c|}{アルゴリズム} & \multicolumn{3}{c|}{コスト} \\ \cline{1-5}
重み  & 探索する順番 & ${\rm W_{1}}$ & ${\rm W_{2}}$ & ${\rm W_{1}}$ \\ \hline
$p_1$ & $t_1, t_2, t_3$ & $a_1$  & $b_1+a_2$  & $b_1+b_2+a_3$  \\ \hline
$p_2$ & $t_2, t_3, t_1$ & $a_1+b_2+b_3$  & $a_2$  & $b_2+a_3$  \\ \hline
$p_3$ & $t_3, t_1, t_2$ & $a_1+b_3 $ & $b_1+a_2+b_3$  & $a_3 $ \\ \hline
$p_4$ & $t_2, t_1, t_3$ & $a_1+b_2$ & $a_2$  & $b_1+b_2+a_3$  \\ \hline
$p_5$ & $t_1, t_3, t_2$ & $a_1$ & $b_1+a_2+b_3$ & $b_1+a_3$ \\ \hline
$p_6$ & $t_3, t_2, t_1$ & $a_1+b_2+b_3$  & $a_2+b_3$ & $a_3$  \\ \hline
\end{tabular}
\end{center}

\begin{flushleft}
  このときのコスト期待値${\rm W_{1}},~{\rm W_{2}},~{\rm W_{3}}~は$
\end{flushleft}

\begin{center}

${\rm W_{1}}=a_1+(b_2+b_3)p_2+b_3p_3+b_2p_4+(b_2+b_3)p_6 \ \ \ \ \, (1.1)$\\
${\rm W_{2}}=b_1p_1+a_2+(b_1+b_3)p_3+(b_1+b_3)p_5+b_3p_6 \ \ \ \ (1.2)$\\
${\rm W_{3}}=(b_1+b_2)p_1+b_2p_2+a_3+(b_1+b_2)p_4+b_1p_5 \ \ \ \ (1.3)$

\end{center}

\begin{flushleft}
  と表すことができる\\
  また、ここでいくつかの主張を示す
\end{flushleft}

\begin{itembox}[l]{主張1-1}
${\rm W_{1}} = {\rm W_{2}}\Leftrightarrow $
$p_2 = \frac{1}{b_1+b_2+b_3}\{-(b_1+b_2)p_4+b_3p_5-(b_1+b_2)p_6\}+\frac{1}{b_1+b_2+b_3}(a_2+b_1-a_1)$
\end{itembox}

(証明)\\
${\rm W_{1}}={\rm W_{2}}\Rightarrow $
$p_2=\frac{1}{b_1+b_2+b_3}\{-(b_1+b_2)p_4+b_3p_5-(b_1+b_2)p_6\}+\frac{1}{b_1+b_2+b_3}(a_2+b_1-a_1)$を示す。
\begin{eqnarray*}
{\rm W_{1}}&=&{\rm W_{2}}\\
a_1+(b_2+b_3)p_2+b_3p_3+b_2p_4+(b_2+b_3)p_6&=&b_1p_1+a_2+(b_1+b_3)p_3+(b_1+b_3)p_5+b_3p_6\\
-b_1p_1+(b_2+b_3)p_2-b_1p_3&=&-b_2p_4+(b_1+b_3)p_5-b_2p_6+a_2-a_1\\
-b_1(p_1+p_3)+(b_2+b_3)p_2&=&-b_2p_4+(b_1+b_3)p_5-b_2p_6+a_2-a_1\\
-b_1(1-p_2-p_4-p_5-p_6)+(b_2+b_3)p_2&=&-b_2p_4+(b_1+b_3)p_5-b_2p_4+a_2-a_1\\
(b_1+b_2+b_3)p_2&=&-(b_1+b_2)p_4+b_3p_5-(b_1+b_2)p_6+a_2+b_1-a_1
\end{eqnarray*}\\


\vspace{5mm}
次に\\
$p_2=\frac{1}{b_1+b_2+b_3}\{-(b_1+b_2)p_4+b_3p_5-(b_1+b_2)p_6\}+\frac{1}{b_1+b_2+b_3}(a_2+b_1-a_1)\Rightarrow $
${\rm W_{1}}={\rm W_{2}}$を示す\\
\vspace{3mm}\\
$p_2=\frac{1}{b_1+b_2+b_3}\{-(b_1+b_2)p_4+b_3p_5-(b_1+b_2)p_6\}+\frac{1}{b_1+b_2+b_3}(a_2+b_1-a_1)$を(1.1),(1.2)に代入すると
\begin{eqnarray*}
{\rm W_{1}}&=&a_1+(b_2+b_3)p_2+b_3p_3+b_2p_4+(b_2+b_3)p_6 \\
&=&a_1+(b_2+b_3) \left(\frac{1}{b_1+b_2+b_3}\{-(b_1+b_2)p_4+b_3p_5-(b_1+b_2)p_6\}+\frac{1}{b_1+b_2+b_3}(a_2+b_1-a_1)\right) \\
&&\hspace{265pt}+b_3p_3+b_2p_4+(b_2+b_3)p_6 \\
&=& a_1+ b_3p_3+\left(\frac{-(b_2+b_3)(b_1+b_2)}{b_1+b_2+b_3}+b_2 \right) p_4 +\frac{b_3(b_2+b_3)}{b_1+b_2+b_3}p_5 \\
&&\hspace{120pt}+ \left(-\frac{(b_2+b_3)(b_1+b_2)}{b_1+b_2+b_3}+(b_2+b_3) \right)p_6+\frac{(b_2+b_3)(a_2+b_1-a_1)}{b_1+b_2+b_3}\\
&=&a_1+b_3p_3-\frac{b_1b_3}{b_1+b_2+b_3}p_4+\frac{b_3(b_2+b_3)}{b_1+b_2+b_3}p_5+
\frac{b_3(b_2+b_3)}{b_1+b_2+b_3}p_6+\frac{(b_2+b_3)(a_2+b_1-a_1)}{b_1+b_2+b_3} \\
&=&\frac{a_1b_1+a_2b_2+a_2b_3+b_1b_2+b_1b_3}{b_1+b_2+b_3}+b_3p_3-\frac{b_1b_3}{b_1+b_2+b_3}p_4+\frac{b_3(b_2+b_3)}{b_1+b_2+b_3}p_5+
\frac{b_3(b_2+b_3)}{b_1+b_2+b_3}p_6
\end{eqnarray*}
\begin{eqnarray*}
{\rm W_{2}}&=&b_1p_1+a_2+(b_1+b_3)p_3+(b_1+b_3)p_5+b_3p_6\\
&=&b_1(p_1+p_3)+a_2+b_3p_3+(b_1+b_3)p_5+b_3p_6 \\
&=&b_1(1-p_2-p_4-p_5-p_6)+a_2+b_3p_3+(b_1+b_3)p_5+b_3p_6\\
&=&b_1+a_2-b_1p_2+b_3p_3-b_1p_4+b_3p_5+(-b_1+b_3)p_6\\
&=&b_1+a_2-b_1\left(\frac{1}{b_1+b_2+b_3}\{-(b_1+b_2)p_4+b_3p_5-(b_1+b_2)p_6\}+\frac{1}{b_1+b_2+b_3}(a_2+b_1-a_1)\right)\\
&&\hspace{230pt}+b_3p_3-b_1p_4+b_3p_5+(-b_1+b_3)p_6\\
&=&b_1+a_2-\frac{b_1(a_2+b_1-a_1)}{b_1+b_2+b_3}+b_3p_3+\left(\frac{b_1(b_1+b_2)}{b_1+b_2+b_3}-b_1 \right) p_4+\left(-\frac{b_1b_3}{b_1+b_2+b_3}+b_3\right)p_5\\
&&\hspace{230pt}+\left(\frac{b_1(b_1+b_2)}{b_1+b_2+b_3}+(-b_1+b_3) \right) p_6\\
&=&\frac{a_1b_1+a_2b_2+a_2b_3+b_1b_2+b_1b_3}{b_1+b_2+b_3}+b_3p_3-\frac{b_1b_3}{b_1+b_2+b_3}p_4+\frac{b_3(b_2+b_3)}{b_1+b_2+b_3}p_5+\frac{b_3(b_2+b_3)}{b_1+b_2+b_3}p_6
\end{eqnarray*}
よって, ${\rm W_{1}}={\rm W_{2}}$\\
以上より、主張が言えた\\
同様の方法で以下の主張1-2,~1-3も示すことができる\\

\vspace{5mm}

\begin{itembox}[l]{主張1-2}
${\rm W_{2}}={\rm W_{3}}\Leftrightarrow p_3=\frac{1}{b_1+b_2+b_3}\{b_1p_4-(b_2+b_3)p_5-(b_2+b_3)p_6\}+\frac{1}{b_1+b_2+b_3}(a_3+b_2-a_2)$
\end{itembox}
%主張3
\begin{itembox}[l]{主張1-3}
${\rm W_{3}}={\rm W_{1}}\Leftrightarrow p_1=\frac{1}{b_1+b_2+b_3}\{-(b_1+b_3)p_4-(b_1+b_3)p_5+b_2p_6\}+\frac{1}{b_1+b_2+b_3}(a_1+b_3-a_3)$
\end{itembox}

また、以上の主張1-1,~1-2,~1-3より以下の主張1-4を導き出せる

\begin{itembox}[l]{主張1-4}
${\rm W_{1}}={\rm W_{2}}={\rm W_{3}} $
{\scriptsize \[\Leftrightarrow
\left(
\begin{array}{c}
p_1 \\
p_2 \\
p_3
\end{array}
\right)
=
\frac{1}{b_1+b_2+b_3}
\left(
\begin{array}{ccc}
-(b_1+b_3) & -(b_1+b_3) & b_2 \\
-(b_1+b_2) & b_3 & -(b_1+b_2) \\
b_1 & -(b_2+b_3) & -(b_2+b_3)
\end{array}
\right)
\left(
\begin{array}{c}
p_4 \\
p_5 \\
p_6
\end{array}
\right)
+
\frac{1}{b_1+b_2+b_3}
\left(
\begin{array}{c}
a_1+b_3-a_3 \\
a_2+b_1-a_1 \\
a_3+b_2-a_2
\end{array}
\right)
\]}
\end{itembox}
さらに、主張1-1,~1-2,~1-3~より以下の主張が導き出せる\\

\begin{itembox}[l]{主張2-1}
${\rm W_{1}} \geqq {\rm W_{2}}\Leftrightarrow $
$p_2 \geqq \frac{1}{b_1+b_2+b_3}\{-(b_1+b_2)p_4+b_3p_5-(b_1+b_2)p_6\}+\frac{1}{b_1+b_2+b_3}(a_2+b_1-a_1)$
\end{itembox}

\begin{itembox}[l]{主張2-2}
${\rm W_{2}}\geqq{\rm W_{3}}\Leftrightarrow p_3\geqq\frac{1}{b_1+b_2+b_3}\{b_1p_4-(b_2+b_3)p_5-(b_2+b_3)p_6\}+\frac{1}{b_1+b_2+b_3}(a_3+b_2-a_2)$
\end{itembox}
%主張3
\begin{itembox}[l]{主張2-3}
${\rm W_{3}}\geqq{\rm W_{2}}\Leftrightarrow p_1\geqq\frac{1}{b_1+b_2+b_3}\{-(b_1+b_3)p_4-(b_1+b_3)p_5+b_2p_6\}+\frac{1}{b_1+b_2+b_3}(a_1+b_3-a_3)$
\end{itembox}

\begin{itembox}[l]{主張2-4}
${\rm W_{2}}\geqq{\rm W_{3}}\Leftrightarrow p_5\geqq\frac{1}{b_1+b_2+b_3}\{-(b_2+b_1)p_1+b_3p_2-(b_2+b_1)p_3\}+\frac{1}{b_1+b_2+b_3}(a_1+b_2-a_2)$
\end{itembox}

\begin{itembox}[l]{主張2-5}
${\rm W_{3}}\geqq{\rm W_{2}}\Leftrightarrow p_4\geqq\frac{1}{b_1+b_2+b_3}\{-(b_3+b_2)p_1-(b_3+b_2)p_2+b_1p_3\}+\frac{1}{b_1+b_2+b_3}(a_2+b_3-a_3)$
\end{itembox}

\begin{itembox}[l]{主張2-6}
${\rm W_{1}}\geqq{\rm W_{3}}\Leftrightarrow p_6\geqq\frac{1}{b_1+b_2+b_3}\{b_2p_1-(b_1+b_3)p_2-(b_1+b_3)p_3\}+\frac{1}{b_1+b_2+b_3}(a_3+b_3-a_1)$
\end{itembox}

\vspace{4mm}

以上の主張を踏まえて、$d_0(T_3)=\Psi_3$を示していく


\newpage


まず、主張1-4$~の右辺を満たすアルゴリズム全体の集合をP$と置く。\\
ここで、$任意のアルゴリズムA\in Pの下では、{\rm W_{1}}={\rm W_{2}}={\rm W_{3}}$となるため\\
${\rm W_{1}},~{\rm W_{2}},~{\rm W_{3}}~のそれぞれに$主張1-4右辺の$p_1,~p_2,~p_3を代入すると$\\
\begin{center}
  ${\rm W_{1}}={\rm W_{2}}={\rm W_{3}}=\displaystyle\frac{a_1b_1+a_2b_2+a_3b_3+b_1b_2+b_2b_3+b_3b_1}{b_1+b_2+b_3}=\Psi_3$
\end{center}
となる\\
\vspace{3mm}\\
また、主張1-4$の右辺を満たさないアルゴリズム全体の集合をQと置く。$\\
$任意のアルゴリズムB\in Qの下で、{\rm W_{1}},~{\rm W_{2}},~{\rm W_{3}}~は、以下の大小関係のいずれかで表される$
\vspace{5mm}
\begin{center}
  ${\rm W_{1}}~\geqq~{\rm W_{2}}~\geqq~{\rm W_{3}}$\\
  ${\rm W_{2}}~\geqq~{\rm W_{3}}~\geqq~{\rm W_{1}}$\\
  ${\rm W_{3}}~\geqq~{\rm W_{1}}~\geqq~{\rm W_{2}}$\\
  ${\rm W_{1}}~\geqq~{\rm W_{3}}~\geqq~{\rm W_{2}}$\\
  ${\rm W_{2}}~\geqq~{\rm W_{1}}~\geqq~{\rm W_{3}}$\\
  ${\rm W_{3}}~\geqq~{\rm W_{2}}~\geqq~{\rm W_{1}}$
\end{center}

ここで、それぞれの大小関係に対して、$\Psi_3~と{\rm W_{1}},~{\rm W_{2}},~{\rm W_{3}}との関係を考える$\\

$\underline{case1}~~~{\rm W_{1}}~\geqq~{\rm W_{2}}~\geqq~{\rm W_{3}}~のとき$\\
\vspace{3mm}
${\rm W_{1}}~\geqq~{\rm W_{2}},~{\rm W_{2}}~\geqq~{\rm W_{3}}であるから、主張1,2より、$
\begin{center}
$p_2 \geqq \frac{1}{b_1+b_2+b_3}\{-(b_1+b_2)p_4+b_3p_5-(b_1+b_2)p_6\}+\frac{1}{b_1+b_2+b_3}(a_2+b_1-a_1)$\\
$p_3\geqq\frac{1}{b_1+b_2+b_3}\{b_1p_4-(b_2+b_3)p_5-(b_2+b_3)p_6\}+\frac{1}{b_1+b_2+b_3}(a_3+b_2-a_2)~~~$
\end{center}
が成立、よって
\begin{center}
  \begin{eqnarray*}
      {\rm W_{1}}&=&a_1+(b_2+b_3)p_2+b_3p_3+b_2p_4+(b_2+b_3)p_6\\
      &\geqq&a_1+\frac{b_2+b_3}{b_1+b_2+b_3}\{-(b_1+b_2)p_4+b_3p_5-(b_1+b_2)p_6\}+\frac{1}{b_1+b_2+b_3}(a_2+b_1-a_1)\\
      &~~&+\frac{b_3}{b_1+b_2+b_3}\{b_1p_4-(b_2+b_3)p_5-(b_2+b_3)p_6\}+\frac{1}{b_1+b_2+b_3}(a_3+b_2-a_2)+b_2p_4+(b_2+b_3)p_6\\\\
      &=&\frac{a_1b_1+a_2b_2+a_3b_3+b_1b_2+b_2b_3+b_3b_1}{b_1+b_2+b_3}\\
      &=&\Psi_3
  \end{eqnarray*}\\
\end{center}
同様に、\\
$\underline{case2}~~~{\rm W_{2}}~\geqq~{\rm W_{3}}~\geqq~{\rm W_{1}}~のとき$\\
\begin{center}
  ${\rm W_{2}}~\geqq~\Psi_3$
\end{center}
$\underline{case3}~~~{\rm W_{3}}~\geqq~{\rm W_{1}}~\geqq~{\rm W_{2}}~のとき$\\
\begin{center}
  ${\rm W_{3}}~\geqq~\Psi_3$
\end{center}
$\underline{case4}~~~{\rm W_{1}}~\geqq~{\rm W_{3}}~\geqq~{\rm W_{2}}~のとき$\\
\begin{center}
  ${\rm W_{1}}~\geqq~\Psi_3$
\end{center}
$\underline{case5}~~~{\rm W_{2}}~\geqq~{\rm W_{1}}~\geqq~{\rm W_{3}}~のとき$\\
\begin{center}
  ${\rm W_{2}}~\geqq~\Psi_3$
\end{center}
$\underline{case6}~~~{\rm W_{3}}~\geqq~{\rm W_{2}}~\geqq~{\rm W_{1}}~のとき$\\
\begin{center}
  ${\rm W_{3}}~\geqq~\Psi_3$
\end{center}

以上より、$任意のB\in Q~に対して、~$max$\{{\rm W_{1}},~{\rm W_{2}},~{\rm W_{3}}\}~\geqq~\Psi_3~$となるため
\begin{center}
  $d_0(T_3)=\Psi_3$
\end{center}


\newpage

$d_0(T_n) = \Psi_n$を示す\\

まず、$d_0(T_n) \leqq \Psi_n$~を帰納法で示す\\

\underline{$k=2のとき$}
\begin{center}
  $d_0(T_2)=\Psi_2$~は既に示されている
\end{center}
\underline{$k=n-1のとき$}
\begin{center}
  $d_0(T_{n-1})=\Psi_{n-1}$~が成立すると仮定する
\end{center}

最初または最後に$t_i~(1,~\ldots,~i,~\ldots,~n)$を調べるアルゴリズム全体の集合を$V$とする
ここで、$T_nをt_iと残りn-1個の部分木の2分木だと捉えると$←書き方わからん

\begin{center}
\begin{tabular}{|c|c|c|c|}
\hline
     & \multicolumn{1}{c|}{アルゴリズム} & \multicolumn{2}{c|}{コスト} \\ \cline{1-4}
重み  & 探索する順番 & ${\rm W_{1}}$ & ${\rm W_{2}}$ \\ \hline
$p_1$ & $t_1, t_2$ & $\Psi_{n-1}$  & $b_1+\cdots+b_{n-1}+a_n$  \\ \hline
$1-p_1$ & $t_2, t_1$ & $b_1+\Psi_{n-1}$  & $a_n$ \\ \hline
\end{tabular}
\end{center}
このときのコスト期待値${\rm W_{1}},~{\rm W_{2}}は$
\begin{eqnarray*}
  {\rm W_{1}}&=&p_1\Psi_{n-1}+(1-p_1)(b_n+\Psi_{n-1})\\
  &=&(1-p_1)b_n+\Psi_{n-1}\\
  {\rm W_{2}}&=&p_1(b_1+\cdots+b_{n-1}+a_n)+(1-p_1)a_n\\
  &=&p_1(b_1+\cdots+b_{n-1})+a_n
\end{eqnarray*}
ここで、${\rm W_{1}}={\rm W_{2}}とすると$
\begin{eqnarray*}
  {\rm W_{1}}&=&{\rm W_{2}}\\
  (1-p_1)b_n+\Psi_{n-1}&=&p_1(b_1+\cdots+b_{n-1})+a_n\\
  p_1(b_1+\cdots+b_{n})&=&\Psi_{n-1}-a_n+b_n\\
  p_1&=&\displaystyle\frac{\Psi_{n-1}-a_n+b_n}{b_1+\cdots+b_{n}}
\end{eqnarray*}
${\rm W_{1}},~{\rm W_{2}}~に~p_1~を代入すると$
\begin{center}
  ${\rm W_{1}}={\rm W_{2}}=\Psi_n$
\end{center}
以上から、$d_0(T_n) \leqq \Psi_n$

\newpage

次に、$d_0(T_n) \geqq \Psi_n$~を示す~~~$W_iのWeight?を~\displaystyle\frac{b_i}{\displaystyle\sum_{j=1}^{n} b_j}$とし\\
\begin{eqnarray*}
cost&=&\displaystyle\frac{b_1}{\displaystyle\sum_{j=1}^{n} b_j}a_1+\displaystyle\frac{b_2}{\displaystyle\sum_{j=1}^{n} b_j}(b_1+a_2)+\cdots+\displaystyle\frac{b_n}{\displaystyle\sum_{j=1}^{n} b_j}(b_1+b_2+\cdots+b_{n-1}+a_n)\\
&=&\displaystyle\frac{1}{\displaystyle\sum_{j=1}^{n} b_j}(a_1b_1+(a_2b_2+b_1b_2)+\cdots+(a_nb_n+b_1b_n+\cdots+b_{n-1}b_n))\\
&=&\displaystyle\frac{\displaystyle\sum_{i=1}^{n}a_ib_i~+~\displaystyle\sum_{i=1,j=1}^{n}a_ib_j}{\displaystyle\sum_{i=1}^{n} b_i}\\
&=&\Psi_n
\end{eqnarray*}\\
よって、Yaoなんちゃらを使って下からおさえる
\end{document}
